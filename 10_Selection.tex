\chapter{Event Selection}
\label{sec:Selection}

The event selection is one of the main phases of the analysis, which is focused on suppressing of as much as possible background events while keeping the signal events. The process of selecting the suitable events goes in three stages. The first stage is the online triggers, which, as name suggests, occurs during the data taking, even before the data is written on the storage. The specifies of the trigger work was described in the corresponding section (see Section~\ref{sec:ATLAS_trigger}). The other two stages are the pre-selection and analysis selection (or just selection), and they are specific to our analysis software ZeeD (described later in Section~\ref{sec:ZeeD}), although the division of the selection in such a way is a common practice because of the large amount of data. These two stages would be described in the following sections.

\section{Analysis preselection}
\label{sec:Sel_pre-sel}

The goal of the preselection stage is to reduce the amount of data (by removing "non-interesting" events) while not biasing the samples in regards of the \Zee\ analysis. There are two flavors of preselection: the two-electron preselection and the one-electron. The cut for the two-electron stream requires $\pt > 14$ GeV for both electrons and the cut is $\pt > 20$ GeV for one-electron. For the central-forward \Zee\ analysis the single-electron stream is mostly used, but the di-electron is also used in some cases, for instance for the trigger efficiency scale factors calculation (tag\&probe method). Such cuts do not affect any possible distribution within the \Zee\ analysis, be it the signal or the background studies. During the preselection stage we also extract all the data that is relevant to the analysis, while disregarding all the rest of the data stream (e.g. muons, jets and so on). This allows us to reduce the physical amount of data from terabytes to just mere gigabytes, and to speed-up the analysis by the factor of hundreds. The resulting pre-selected data is very similar to the D3PD from the programming point of view (which is a commonly used data format that was developed after AOD, and was later incorporated in AOD successor - xAOD), but is designed specifically for the needs of \Zee\ analysis (see Section~\ref{sec:ZeeD_TTrees} for details).

\section{Analysis cuts}
\label{sec:Sel_cuts}

The application of cuts is one of the most important parts of the analysis. During this stage we try to suppress the background while keeping the signal intact. The cuts can be divided in three categories: the technical cuts, the kinematic cuts and the electron "goodness" cuts. The technical cuts are applied on a per-event basis. Here we check that no problems were encountered during the event taking. This includes a so-called good run list, which excludes the events taken during the runs with observed problems in the detector, and the OQ-map (object quality map) which is a list of a regions of the calorimeter which are known to be faulty, and so the cut excludes all the events with at least one electron depositing energy in one of those regions.

The kinematic cuts deal with the kinematic properties of the electrons: we require only two electrons, with one being inside the central region and one inside the forward region, both are reconstructed with the proper reconstruction algorithm and both meeting the selection requirements for the kinematic properties.

Finally there are cuts on the quality of the electrons, which involve passing certain IsEM criteria and a certain isolation criteria.

Every group of cuts will be described in details.

\subsection{Data quality cuts}
\label{sec:Sel_GRL_OQ}

\begin{itemize}
\item {\bfseries Good Run List}: This cut drops every event that was taken during the non-successful runs of the LHC. The list of the good runs is compiled by the Data Quality group and is the same for all analyses.
\item {\bfseries Object Quality Maps}: This cut drops every electron that was reconstructed inside one of the faulty regions of the calorimeter. The events with such electrons can still be used in the analysis, if they have another two good electrons.
\item {\bfseries LAr Veto}: This cut drops every event that was taken while the LAr calorimeter was malfunctioning, as indicated by the \texttt{LArErrorState} property.
\end{itemize}

\subsection{Kinematic cuts}
\label{sec:Sel_kinematic}

\begin{itemize}
\item {\bfseries Two good electrons per event}: This cut drops all events that have the number of good electrons other then two. The Drell-Yan process produces exactly two electrons, so everything else is a background for our analysis. In case of central-forward analysis, the cut can be renamed as {\bfseries 1+1 good electrons per event}, as we require exactly one good electron in the central region, and one in the forward region.
\item {\bfseries Minimum electron \pt}: This cut drops all electrons with the $\pt < 20$~GeV. We do not have adequate efficiency corrections for the electrons with low \pt.
\item {\bfseries Electron $|\eta|$}: This cut drops all electrons which have a cluster that is reconstructed outside the required $\eta$ regions: $|\eta| < 2.47$ for the central electron and $|\eta| > 2.52$ for the forward electron.
\item {\bfseries Electrons are outside the crack regions}: This cut drops all electrons reconstructed inside the crack regions of the calorimeters. For the central electrons the cracks are $1.47 < |\eta| < 1.52$ and $|\eta| > 2.47$, for the forward electrons it is $3.16 < |\eta| < 3.35$
\item {\bfseries Z boson mass}: This cut drops all events with the di-electron mass outside the mass window range ($66 < |M_{ee}| < 116$~GeV for peak mass and $116 < |M_{ee}| < 150$~GeV for high mass regions).
\item {\bfseries Number of tracks at primary vertex}: This cut drops all events that do not have a vertex with at least three tracks.
\end{itemize}

\subsection{Electron quality cuts}
\label{sec:Sel_isem_iso}

\begin{itemize}
\item {\bfseries A track for the central electron}: This cut drops all events with the central electron having no track. The "author" cut which requires the specific author for every electron supersedes this cut, but it is kept for compatibility purposes.
\item {\bfseries Central electron IsEM}: This cut requires the central electron to satisfy the IsEM criteria (described in Section~\ref{sec:Rec_elecID}).
\item {\bfseries Forward electron IsEM}: This cut requires the forward electron to satisfy the IsEM criteria.
\item {\bfseries Central electron isolation}: This cut requires the central electron to satisfy the isolation criteria (described in Section~\ref{sec:Rec_eleciso}).
\item {\bfseries Electron author}: This cut requires the proper author (described in Section~\ref{sec:Rec_elec}) for both central and forward electron.
\end{itemize}

The cutflows, i.e. the tables that shows how many events pass any given cut, can be seen in Tab.~\ref{tab:sel_cutflow_data} for data and in Tab.~\ref{tab:sel_cutflow_MC} for the signal MC sample. The relative efficiency shows how many events pass the cut relative to the previous cut, while the absolute efficiency shows the number of events relative to the total number of events.

\begin{table}
\centering
\begin{tabular}{@{}lrrr@{}} \hline \hline
 Cut & events & $\epsilon_{rel}$ [\%] & $\epsilon_{abs}$ [\%] \\\hline
 All events (after pre-selection)  &    234255443 & &   100.00000 \\
 primary vertex w. $>2$ tracks &    234127638 &     99.9454 &    99.94544 \\
 veto LAr noise bursts & 233365291 &     99.6744 &    99.62001 \\
$|\eta_{cnt}| < 2.47$, $2.5<|\eta_{fwd}|<4.9$ &    138946695 &     59.5404 &    59.31418 \\
excl. $1.47 < |\eta| < 1.52$ &    127877309 &     92.0334 &    54.58883 \\
excl. $3.16 < |\eta| < 3.35$ &    124775277 &     97.5742 &    53.26462 \\
$pt_{e}^{cent.} > 25$\,GeV &    74478333 &     59.6900 &    31.79364 \\
$pt_{e}^{fwd.} > 20$\,GeV &      8404968 &     11.2851 &     3.58795 \\
Author &     8404244 &     99.9914 &     3.58764 \\
good object quality &     8338286 &     99.2152 &     3.55948 \\
 Tight++ &      1537446 &     18.4384 &     0.65631 \\
 FwdTight &       370376 &     24.0903 &     0.15811 \\
 Iso98Etcone20 &       355029 &     95.8564 &     0.15156 \\
 Iso97Ptcone40 &       340852 &     96.0068 &     0.14550 \\
 single-lepton trigger &       339235 &     99.5256 &     0.14481 \\
 MaxTwoGoodElectrons &       339235 &    100.0000 &     0.14481 \\\hline
 $66 < m_{ee} < 116$~GeV &       321575 &     94.7942 &     0.13728 \\
 $116 < m_{ee} < 150$~GeV &         7740 &        2.28 &     0.0033 \\ \hline \hline
\end{tabular}
\caption{The cutflow for the \Zee\ CF data selection both for peak and high mass windows.}
\label{tab:sel_cutflow_data}
\end{table}

\begin{table}
\centering
\begin{tabular}{@{}lrrr@{}} \hline \hline
 Cut & events (weighted)  & $\epsilon_{rel}$ [\%] & $\epsilon_{abs}$ [\%] \\\hline
 All events (after pre-selection) &     19783527.83 &             &   100.00000 \\
 primary vertex w. $>2$ tracks &     19659928.15 &     99.3752 &    99.37524 \\
 veto LAr noise bursts &     19659928.15 &    100.0000 &    99.37524 \\
$|\eta_{cnt}| < 2.47$, $2.5<|\eta_{fwd}|<4.9$ &     12425328.01 &     63.2013 &    62.80643 \\
excl. $1.47 < |\eta| < 1.52$ &     11837779.90 &     95.2714 &    59.83655 \\
excl. $3.16 < |\eta| < 3.35$ &     11431076.75 &     96.5644 &    57.78078 \\
$pt_{e}^{cent.} > 25$\,GeV &      7959710.26 &     69.6322 &    40.23403 \\
$pt_{e}^{fwd.} > 20$\,GeV &      3207445.87 &     40.2960 &    16.21271 \\
Author &      3206917.09 &     99.9835 &    16.21004 \\
good object quality &      3186720.53 &     99.3702 &    16.10795 \\
 Tight++ &      2228171.55 &     69.9205 &    11.26276 \\
 FwdTight &      1684155.81 &     75.5847 &     8.51292 \\
 Iso98Etcone20 &      1661813.91 &     98.6734 &     8.39999 \\
 Iso97Ptcone40 &      1629799.05 &     98.0735 &     8.23816 \\
 single-lepton trigger &      1584699.06 &     97.2328 &     8.01019 \\
 MaxTwoGoodElectrons &      1584699.06 &    100.0000 &     8.01019 \\ \hline
 $66 < m_{ee} < 116$~GeV &      1542376.88 &     97.3293 &     7.79627 \\
 $116 < m_{ee} < 150$~GeV &        42206.05 &        2.66 &     0.21 \\ \hline \hline
\end{tabular}
\caption{The cutflow for the \Zee\ CF MC selection both for peak and high mass windows. The number of events is corrected for weights.}
\label{tab:sel_cutflow_MC}
\end{table}
