\chapter{Background estimation}
\label{sec:Bkg}

As was described in Sec.~\ref{sec:Selection}, the selection phase is designed to suppress the background events (i.e. not \Zee\ events) while keeping the signal events. But even with all the cuts applied, some of the background events still pass all of them. In order to get the correct results, we need to estimate the number of the background events in our selection.

The background events come in two flavours: the electroweak (EW) background and the QCD background. The difference from the analysis point of view is in that we can directly predict the amount of the EW background based on the MC simulation, while the QCD background we have to estimate, using so-called fits based on the indirect data. Both of these methods will be descrabed here.

\section{Electroweak background}

The source of the electroweak background is the events that came from various electroweak decays but were misinterpreted as \Zee. The list of the processes that contribute to the EW background was given in Tab.~\ref{tab:MC_bg}. There are seven of them, including three single-boson decays, three di-boson decays, and \ttbar. All the processes have a cross-section comparable with \Zee\ which can be reliably predicted by the theory, and also use the same effiviency coefficients as signal, and therefore can be simulated using MC.

\section{QCD background}
