\chapter{Event Selection}
\label{sec:Selection}

The event selection is one of the main phases of the analysis, during it we try to suppress as much as possible background events while keeping the signal events. The process of selecting the suitable events goes in three stages. The first stage is the online triggers, which, as name suggests, occurs during the data taking, even before the data is written on the storage. The other two stages are specific to our analysis software ZeeD (described later in Sec.~\ref{sec:ZeeD}), and even the division to pre-selection and actual selection is due to some specifics of ZeeD (described in Sec.~\ref{sec:ZeeD_TTrees}). Every stage will be described here in the corresponding section.

\section{Online triggers}
\label{sec:Sel_trig}

The online triggers are the part of the standard ATLAS data-taking procedure. It consist of two trigger layers (the so-called L1 trigger and L2 trigger), and one event filter~\cite{lib:trigger}. The frequency at which LHC delivers bunch collisions is 40 MHz (every 25 ns). It's physically impossible to reconstruct and record every of these 40 millions events per second. And what's more important, only a fraction of these events contain the actual $pp$ collision, so most of the events would be useless even if we managed to record them all. The gradual sistem of triggers makes an on-the-fly decision whether the event has an actual $pp$ collision or not. The L1 trigger works on the frequency of LHC bunches crossing. This trigger is fully hardware, so it takes a decision in a matter of nanoseconds during the time while the signal from the detectors travels the cables to the readout drivers (RODs). The next level trigger is already software based, it is called L2 trigger, it works on the frequency of 60 KHz which gives it about 15 $\mu$s to make a decision. It's output event rate is 5 KHz. The final stage is an event filter which outputs the events at the average rate of 400 Hz. At this rate the events are saved to the output streams. The L2 trigger and the event filter are also collectively called the High Level Trigger (HLT), since they study the event much deeply then the L1 trigger, which only measures the total amount of $E_{t}$ in the calorimeter towers with the precision of 1 GeV. The L2 trigger attempts a partial reconstruction of the event in the regions of interest (the calorimeter towers found by L1) as well as track reconstruction, and event filter does the full event reconstruction in these regions.

\section{Analysis pre-selection}
\label{sec:Sel_pre-sel}

The goal of the preselection stage is to reduce the amount of the data while not biasing the samples in regards of the \Zee\ analysis. There are two flavours of preselection: the two-electron preselection and the one-electron. The cut for the two-electron stream is $P_{t} > 14$ GeV for both electrons and for one-electron the cut is $P_{t} > 20$ GeV. For the central-forward \Zee\ analysis mostly the single-electron stream is used, but the di-electron is also used in some cases, for instance for the trigger efficiency scale factors (tag\&probe method). Such cuts do not affect any possible distribution within the \Zee\ analysis, be it the signal or the background studies. During the preselection stage we also extract all the data that is relevant to the analysis, while disregarding all the rest of the data stream (e.g. muons, jets and so on). This allows us to reduce the phisical amount of data from terabytes to just mere gigabytes, and to speed-up the analysis by the factor of hundreds. The resulting pre-selected data in it's physical form is very similar to the D3PD, but is customly designed specifically for the needs of \Zee\ analysis.

\section{Analysis cuts}
\label{sec:Sel_cuts}

The application of cuts is one of the most important parts of the analysis. During this stage we try to suppress the background while keeping the signal intact. The cuts can be divided in three categories: the technical cuts, the kinematic cuts and the electron "goodness" cuts. The technical cuts are applied on a per-event basis. Here we check that during the event taking no problems were encounterd. This includes a so-called good run list, which excludes the events which were taken during the runs with observed problems in the detector, and the OQ-map (object quality map) which is a list of a regions of the calorimeter which are known to be faulty, and so the cut excludes all the events with at least one electron depositing energy in one of those regions. The kinematic cuts deal with the kinematic properties of the electrons: we require only two electrons, with one being inside the central region and one inside the forward region, both reconstructed with the correct reconstruction algorithm and both meeting the needed criteria for the kinematic properties. The final cuts are the cuts on quality of the elctrons, which involves the meeting of a certain IsEM criteria and a certain isolation criteria. Every group of cuts will now be described fully.

\subsection{Data quality cuts}
\label{sec:Sel_GRL_OQ}

\begin{itemize}
\item {\bfseries Good Run List}:
\item {\bfseries Object Quality Maps}:
\item {\bfseries LArVeto}: \tbu NEEDED?
\end{itemize}

\subsection{Kinematic cuts}
\label{sec:Sel_kinematic}

\begin{itemize}
\item {\bfseries Two electrons per event}:
\item {\bfseries Minimum electron $P_{t}$}:
\item {\bfseries Electron $|\eta|$}:
\item {\bfseries Electrons are outside the crack regions}:
\item {\bfseries Z boson mass}:
\item {\bfseries Number of tracks at primary vertex}:
\item {\bfseries ZeeDCutTrackCentralElecZ}: \tbu NEEDED?
\end{itemize}

\subsection{Electron quality cuts}
\label{sec:Sel_isem_iso}

\begin{itemize}
\item {\bfseries Central electron IsEM}:
\item {\bfseries Forward electron IsEM}:
\item {\bfseries Central electron isolation}: \tbu ONLY CENTRAL?
\item {\bfseries ZeeDCutAuthorCFElecZ}: \tbu WTF IS THIS?
\end{itemize}

\tbu et/pt miss?

cutflow (MC/Data)
