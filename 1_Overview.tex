\chapter{Thesis overview}
\label{sec:TOverview}

\section{Thesis organization}
\label{sec:TOrganization}

The thesis is organized as follows:
\begin{description}
\item Chapter~\ref{sec:Theory}. Brief introduction to the SM, to $pp$ physics and to $Z$ boson production is given. This chapter also provides a motivation for the \Zee cross-section measurement.

\item Chapter~\ref{sec:LHC}. The CERN accelerator complex with LHC is shortly described.

\item Chapter~\ref{sec:ATLAS}. Description of the ATLAS detector and its relevant sub-components is presented. The physics program of the ATLAS experiment is outlined. The ATLAS data acquisition system and computing strategy are discussed as well.

\item Chapter~\ref{sec:DataSamples}. The ATLAS data with their luminosity which were collected in 2011 and used for this thesis are described.

\item Chapter~\ref{sec:MCSamples}. The MC samples obtained with different MC generators are discussed. The full chain of MC production in the ATLAS experiment is shown. A study of the pileup effect and reweighting of the $Z$ boson transverse momentum $p_{T,Z}$ is presented. These studies contribute to paper of the measurement of the $Z$ boson transverse momentum.

\item Chapter~\ref{sec:Reconstruction}. The ATLAS event reconstruction for EM objects, which includes trigger performance, electron reconstruction and identification, is described.

\item Chapter~\ref{sec:Selection}. The event selection for $Z \to e^+e^-$ analysis is discussed. The Central-Forward (CF) selection is introduced.

\item Chapter~\ref{sec:ZeeCrossSec}. The method of the cross-section extraction and the binning definition are presented. Studies of bin-migration effects are performed.

\item Chapter~\ref{sec:ZeeD}. The analysis framework which was developed for presented analysis is described.

\item Chapter~\ref{sec:Calibration}. The method of the off-line calibration of the EM calorimeter is explained and calibration results are presented. The systematic uncertainties of the calibration are discussed as well. Resolution of the EM calorimeter is estimated. Presented calculations contribute to electron performance paper.

\item Chapter~\ref{sec:Efficiency}. The electron trigger, identification, reconstruction and isolation efficiencies are described. Measurements of the isolation and identification efficiencies are performed and comparisons with the official ATLAS prescription are presented.

\item Chapter~\ref{sec:Bkg}. Studies of the background are presented. Several methods for background estimation are discussed.

\item Chapter~\ref{sec:Results}. Inclusive and single differential cross-section measurements in bins of di-electron rapidity are presented. The background estimation, efficiency determination, acceptance measurement and systematic uncertainties are discussed. The differential cross sections in different channels are combined and compared with theoretical predictions. The presented measurement contribute to the ATLAS paper of the inclusive $W$ and $Z$ cross-section measurement.

\item Chapter~\ref{sec:Summary}. A summary of the main results of this thesis is given.
\end{description}
