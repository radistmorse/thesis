\chapter{Thesis overview}
\label{sec:TOverview}

This paper describes the measurement of the neutral current Drell-Yan cross-section in the electron channel measured in the forward region of the ATLAS detector. The data was collected throughout 2011 during the collisions at $\sqrt{s} = 7$~TeV at LHC. For the theoretical predictions, the Monte Carlo (MC) simulation from the MC11c and MC11d campaigns was used. The theoretical predictions are based on various parton distribution functions (PDFs) derived from previous experimental data.

All the details of the cross-section measurements are described in the corresponding chapters. The highlights of the analysis include the details of the Monte Carlo generation, the specificies of the event reconstruction and selection, and the distinct features of the ZeeD analysis software, which allowed to speed-up the analysis process by a factor of hundreds.

The focus of this work was the central-forward (CF) variant of the \Zee\ decay, where "central-forward" means that one of the two resulting electrons is detected in the forward calorimeter, while the other in the central. While the CF analysis share a lot with the central-central (CC) one, which is the other possible variant of the \Zee\ decay (both electrons go to the central parts of the detector), there are still some differences in both the analysis chain and in the results. The combination of the two variants of the analysis is possible, and was done in a paper~\cite{lib:wz2011} that is yet to be published. The work done for this thesis contributed to that paper.


\section{Thesis organization}
\label{sec:TOrganization}

The thesis is organized as follows:
\begin{description}
\item Chapter~\ref{sec:Theory}. Brief introduction to the SM, to $pp$ physics and to $Z$ boson production is given. This chapter also provides a motivation for the \Zee\ cross-section measurement.

\item Chapter~\ref{sec:LHC}. The CERN accelerator complex with LHC is shortly described.

\item Chapter~\ref{sec:ATLAS}. Description of the ATLAS detector and its relevant sub-components is presented. The physics program of the ATLAS experiment is outlined. The ATLAS data acquisition system and computing strategy are discussed as well.

\item Chapter~\ref{sec:DataSamples}. The ATLAS data with its luminosity which was collected in 2011 and used for this thesis is described.

\item Chapter~\ref{sec:MCSamples}. The MC samples obtained with different MC generators are discussed. The full chain of MC production in the ATLAS experiment is shown. A Frozen Showers system which is the fast simulation system is described, which was developed as part of 2011 MC data preparation.

\item Chapter~\ref{sec:Reconstruction}. The ATLAS event reconstruction for EM objects, which includes trigger performance, electron reconstruction and identification, is described.

\item Chapter~\ref{sec:Selection}. The event selection for $Z \to e^+e^-$ analysis is discussed. The Central-Forward (CF) selection is introduced.

\item Chapter~\ref{sec:ZeeCrossSec}. The method of the cross-section extraction and the binning definition are presented. Several extrapolation technics are shown.

\item Chapter~\ref{sec:ZeeD}. The analysis framework which was developed for this analysis is described. The method to compress data and speed-up the calculations is shown.

\item Chapter~\ref{sec:Calibration}. The method of the off-line calibration of the EM calorimeter is explained. The systematic uncertainties of the calibration are discussed as well. Resolution of the EM calorimeter is estimated. Impact of the Fast Simulation technics is discussed.

\item Chapter~\ref{sec:Efficiency}. The electron trigger, identification, reconstruction and isolation efficiencies are described. The specificies of the forward electron efficiencies are discussed.

\item Chapter~\ref{sec:Bkg}. Studies of the background are presented. Several methods for background estimation are discussed.

\item Chapter~\ref{sec:Results}. Inclusive, single and double differential cross-section measurements in bins of di-electron rapidity and mass are presented. The background estimation, efficiency determination, acceptance measurement and systematic uncertainties are discussed.

\item Chapter~\ref{sec:Summary}. A summary of the main results of this thesis is given.
\end{description}
