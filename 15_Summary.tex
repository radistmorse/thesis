\chapter{Summary}
\label{sec:Summary}

This thesis represented the results of the \Zee\ central-forward cross-section measurement based on the $4.6 \mathrm{fb}^{-1}$ of the data collected on the ATLAS detector of the LHC during the 2011 year.

In order to get these results some technical solutions were developed for both ATLAS software infrastructure and the analysis software itself. For the ATLAS software the Frozen Showers system was developed. This system reduced the amount of time needed for the Monte Carlo simulation by about 25\% while introducing a small error that was well within the uncertainties introduced by the technical limitations of the detector. The Frozen Shower system was enabled by default in all the Monte Carlo samples starting with the 2011 for all the analyses. The MC samples used in this analysis were also produced with Frozen Showers enabled.

For the analysis software the new TTree system was developed that was used for data storage. This system reduced the amount of space needed for data samples from terabytes to gigabytes and subsequently allowed to launch the analysis software on local machines and reduced the time needed for the analysis. With the TTree input the ZeeD software was able to process 1000 events per second which makes a couple of hours for the whole dataset, instead of several days that the same task would take while on the GRID reading from the raw AODs. The analysis presented in this thesis was done using the TTree system.

The different stages of the analysis included the identification of the electrons, finding of the boson candidates, filtering of the \Zee\ events based on several criteria (cuts), estimating of the background and the unfolding. Each stage was described in a separate chapter, presenting its impact on the result and the possible sources of the uncertainties it adds. The resulting double-differential \Zee\ central-forward cross-section  together with all sources of uncertainties was presented. This cross-section agrees well with the results of \Zee\ central-central and \Zmm\ analyses, and in combination with them allows for a calculation of PDF fits.
